\documentclass[a4paper,11pt]{article}

\usepackage{amsmath}
\usepackage{amssymb}
\usepackage{amsthm}
\usepackage{graphicx}
\usepackage{enumerate}
\usepackage{xcolor}
\usepackage{biblatex}

\addbibresource{percolation.bib}

%------------------

%\setlength{\topmargin}{0.0in}
%\setlength{\textheight}{10in}
%\setlength{\oddsidemargin}{0.0in}
%\setlength{\evensidemargin}{0.0in}
%\setlength{\textwidth}{6.5in}

%-------------------
\newtheorem{theorem}{Theorem}[section]
\newtheorem{proposition}[theorem]{Proposition}
\newtheorem{lemma}[theorem]{Lemma}
\newtheorem{corollary}[theorem]{Corollary}
\newtheorem{conjecture}[theorem]{Conjecture}


\theoremstyle{definition}
\newtheorem{definition}[theorem]{Definition}
\newtheorem*{example}{Example}
\newcommand{\ints}{\mathbb{Z}}
\newcommand{\sigalg}{$\sigma$-algebra }
\newcommand{\ztwodual}{(\ints^2)^*}
\newcommand{\prob}{\mathbb{P}_p}
%------------------

%Everything before begin document is called the pre-amble and sets out how the document will look
%It is recommended you don't touch the pre-amble until you are familiar with LaTeX

\begin{document}
\thispagestyle{empty}

\begin{figure}[h]
\begin{center}
\includegraphics[scale=0.5]{uob.pdf} %make sure the pdf file named 'uob' is saved in the same folder as this file
\end{center}
\end{figure}

\begin{center}
{\Large Percolation\\ \vspace{1cm}Jonathan Marriott}
\end{center}

\vspace{3cm}
\hrule
\begin{center}
Supervised by Dr Edward Crane\\
Level 6\\
20 Credit Points
\end{center}
\hrule

\vspace{3cm}
\begin{center}
\today
\end{center}
	
\title{Percolation}
\author{Jonathan Marriott}
\date{}
\maketitle

% \begin{abstract}
% Some sorts of documents need abstracts. Others do not.
% \end{abstract}

%The following code is not run because of the percentage sign, but you might find it useful for future work
%  \tableofcontents


\section{Introduction}

A project on Percolation 
%Using the percentage symbol, you can include comments in your code that do not appear in the output.


% \section{Formatting}

% We can \emph{emphasis} some words, i.e., make them \emph{italic}, and we can make some words \textbf{bold}.
% Note how using a new line in the code does not correspond to a new line in the output file.
% Same if we have        a           large                white                   space.

% Instead, if we want a new line/new paragraph, you need to press enter twice, or use \\
% which starts a new line but not a new paragraph.

% \subsection{lists}

% Lists can be numbered or unnumbered, and you can have sub-list inside a list.

% \begin{enumerate}
% 	\item This is the first item in a numbered list.

% 	\item And the second
	
% 	\item 
% 	\begin{enumerate}
% 		\item Here the third item is in fact a numbered sub-list.
% 		\item item 2 of the numbered sub-list
% 	\end{enumerate}

% 	\item 
% 	\begin{itemize}
% 		\item Here the fourth item is an unnumbered sub-list.
% 		\item item 2 of the unnumbered sub-list
% 	\end{itemize}
% \end{enumerate}
\section{The Percolation Model}
\subsection{Initial Definitions}

We start with some basic definitions for Percolation on cubic lattices, specifically bond percolation where we consider the edges on the graph to be either open or closed. 

\begin{definition}\label{ints} 
	$\mathbb{Z} = \{...,-2,-1,0,1,2,...\}$ and $\mathbb{Z}^d = \{(x_1,x_2,...,x_d) : x_i \in \mathbb{Z} \}$
\end{definition}

\begin{definition}\label{dist}
	For $x,y\in \mathbb{Z}^d$, define the distance from x to y, denoted $\delta(x,y)$, by
	$$\delta(x,y) := \sum_{i=1}^{d}|x-y|$$
\end{definition}\label{lattice} 

\begin{definition}[d-dimensional cubic lattice]{
	We construct the lattice with vertices in $\ints^d$ and edges where the distance between vertices is one.
	$$E(\ints^d) = \{\{u,v\}:u,v \in V(\ints^d),\: \delta(u,v) = 1\}$$
	We will often refer to this lattice by the vertex set $\ints^d$ without specifying the edge set.
	We also denote the origin by 0.
}
\end{definition}


\subsection{Probability Space}

We now introduce the Measure theory basics required to define the probability measure and subsequently the probability space for our percolation model.

\begin{definition}[$\sigma$-algebra]
	For some set $X$, we call $\mathcal{A} \subseteq \mathcal{P}(X)$, a subset of the power set of X, a  $\sigma$-algebra of $X$ if:
	\begin{enumerate}
		\item $\emptyset,X \in \mathcal{A}$
		\item $A \in \mathcal{A} \implies A^c := X \setminus A \in \mathcal{A}$ (Closed under complement)
		\item For $A_i \in \mathcal{A}, i \in \mathbb{N}$ we have that $\bigcup^\infty_{i=1} A_i \in \mathcal{A}$ (Closed under countable unions)
	\end{enumerate}
	We call the pair $(X,\mathcal{A})$ a measurable space, and elements of $\mathcal{A}$ measurable sets.
\end{definition}

We note by De Morgan's Laws that a $\sigma$-algebra is also closed under countable intersections.

\begin{definition}[Measure]
	A measure $\mu$ on a measurable space $(X,\mathcal{A})$ is a function $\mu: \mathcal{A} \rightarrow [0,\infty]$ where 
	$\mu(\emptyset) = 0 $ and for disjoint $A_i \in \mathcal{A}, i \in \mathbb{N}$ we have that
	$$ \mu(\bigcup^\infty_{i=1} A_i) = \sum_{i = 1}^{\infty} \mu(A_i)  $$
	We call the triple $(X,\mathcal{A},\mu)$ a measure space.
\end{definition}

In the context above if we think about X being our set of outcomes, and the sigma algebra of X representing the set of events we wish to assign a probability to, it's intuitive to give these events a probability by a measure. 
Clearly we need to restrict our probability measure such that its domain is the interval $[0,1]$ 

\begin{definition}[Probability Measure]
	Let $\Omega$ be the set of all outcomes (the sample space) and $\mathcal{F}$ be a \sigalg of $\Omega$ where the elements are events we wish to consider (the event space). Then a measure $\mathbb{P}$ on $(\Omega,\mathcal{F})$ is a probability measure if:
	\begin{enumerate}
		\item $\mathbb{P}: \mathcal{F} \rightarrow [0,1]$
		\item $\mathbb{P}(\Omega) = 1$
	\end{enumerate}
	Then we call the triple $(\Omega,\mathcal{F},\mathbb{P})$ a probability space.
	
\end{definition}

\subsection{The Model}

We now take some $p \in [0,1]$ which will be our parameter which specifies the probability a given edge is open. 
Setting $q = 1-p$ we say each edge is independently open with probability p and closed with probability q.
We can think of the open and closed edges defining a random subgraph of $\ints^d$ where only edges set to open are retained.\\

We let our sample space $\Omega = \prod_{e \in E} \{0,1\}$ where $E = E(\ints^d)$ and an edge in state 1 represents it is open, and 0 that it is closed.\\
We may refer to $\Omega$ as the set of configurations, where a configuration $\omega$ is a function assigning edges to be open or closed, meaning 
$$\omega:E \to \{0,1\}, e \mapsto \omega_e$$
Then we must define $\mathcal{F}$, some \sigalg of $\Omega$, the events we wish to assign a probability to. 
Clearly we cannot just take $\mathcal{F} = \mathcal{P}(\Omega)$, since we have uncountably many configurations in $\Omega$. 
This can be easily seen via a diagonalization argument. 
It turns out we can generate the \sigalg we want by the cylinder sets of $\Omega$. We base our definition on the one given by \cite{bollo2006}

\begin{definition}[Cylinder Set]
	We say a subset $C \subseteq \Omega$ is a cylinder set if and only if there exists a finite subset $F \subseteq E$ and $\sigma \in \{0,1\}^F$ such that 
	$$C(F,\sigma) = \{\omega \in \Omega: \omega_f = \sigma_f \text{ for }f \in F\}$$
	In essence the cylinder set is the set of configurations $\omega$ which map all the edges in F to the same state that $\sigma$ does.
\end{definition}

Then we define $\mathcal{F}$ to be the set of all unions of cylinder sets of $\Omega$, then $\mathcal{F}$ is said to be generated by the cylinder sets of $\Omega$.
In this case $\mathcal{F}$ is a \sigalg of $\Omega$. Intuitively $\mathcal{F}$ is the set of events which only depend on a finite number of edges.\\
Next we define our probability measure on $(\Omega, \mathcal{F})$ by 
$$\prob(C(F,\sigma)) = \prod_{f \in F} (p(\sigma_f) + q(1-\sigma_f)) $$
Where as usual $q = 1-p$, and we use the subscript $p$ to emphasize that $p$ is the parameter in our model. Then $(\Omega, \mathcal{F}, \prob)$ is our probability space in which we examine the percolation model.
\\
We now lay out some definitions specific for percolation.
\begin{definition}
	Let $C(x)$ denote the open cluster (component) containing x, which is the set of vertices in $\ints^d$ which are connected to x by a path of open edges.
	We abbreviate the open cluster containing the origin $C(0)$ by $C$
\end{definition}

\begin{definition}[Percolation function]{
	We define the percolation function $\theta(p)$ as follows
	$$\theta(p) = \mathbb{P}(|C| = \infty)$$
	In words the percolation function is simply the probability that we can reach an infinite number of vertices from the origin by open edges. 
	We also note that this is the same as asking what is the probability of an having an infinite length self-avoiding path of open edges starting at the origin.
}
	
\end{definition}


% \begin{definition}\label{my_def}
% 	A \emph{label} allows the user to tell Latex 'remember the numbering of that definition/theorem/equation'
% \end{definition}

% \begin{lemma} \label{my_lem}
% 	If something has a label, then we can refer to it, without knowing what number it is 
% \end{lemma}

% \begin{proof}
% 	For example, by calling up Definition \ref{my_def}. This works even if the ordering of things move.
% 	Note that the end of proof square box is already there
% \end{proof}

% \begin{theorem}
% 	And a final theorem
% \end{theorem}

% \begin{proof}
% 	Combining Definition \ref{my_def} with Lemma \ref{my_lem} we get Equation \ref{my_eqn} below.
% \end{proof}

\section{Existence of a critical value}
\subsection {Existence of a critical value on $\ints$}\label{critvalforZ}
Trivially the critical value is p = 1. Consider the event $X_n = \{$There is an open self-avoiding path of length n starting at the origin$\}$ 
Then $X_n \supseteq X_{n+1}$ and so 
$$\lim_{n\rightarrow \infty} \mathbb{P} (X_n) = \theta(p)$$
And since $\mathbb{P} (X_n) = 2p^n$, as the path can go left or right from the origin. We have for all $p <1$, $\theta(p) = 0$. Thus, $\theta(p) > 0$ if and only if $p = 1$. 


\subsection {Existence of a critical value on $\ints^2$}
We show the existence of the critical value in this case by bounding it from above and below. We follow the proofs given in \cite{steif2011mini}
\begin{theorem}
	If $p < 1/3$, $\theta(p) = 0$.
\end{theorem}
\begin{proof}
	Let $X_n = \{$There is an open self-avoiding path of length n starting at the origin$\}$ as in Section \ref{critvalforZ}.
	Then the probability for a path of length n to be open on every edge is $p^n$. The number of paths of length n from the origin is at most $4(3^{n-1})$ since there are 4 edges to choose from at the origin, then for each next step in the path there are at most 3 edges we can pick as the path is self-avoiding.
	Hence, we get $\mathbb{P}(X_n) \leq  4(3^{n-1})p^n  $. Then we take the limit since
	$\lim_{n\rightarrow \infty}\mathbb{P}(X_n) = \theta(p)$.
	\begin{align*}
		\lim_{n\rightarrow \infty}\mathbb{P}(X_n) &\leq  \lim_{n\rightarrow \infty}4(3^{n-1})p^n\\
		 &\leq 4\cdot  3^{-1} \lim_{n\rightarrow \infty}(3p)^{n}
	 \end{align*}
	 Since $p < 1/3$ we have $\theta(p)=\lim_{n\rightarrow \infty}\mathbb{P}(X_n) = 0$
\end{proof}

\begin{theorem}
	For p close to 1, we have $\theta(p) > 0$
\end{theorem}
\begin{proof}
	We introduce the dual graph $(\ints^2)^*$ which has vertices in $(\ints^2 + \binom{1/2}{1/2} )$, and edges as you would expect between vertices at distance 1.
	Then we can see there is a clear correspondence between the edges of $\ints^2$ and its dual, since each edge in the dual intersects a unique edge in the original graph. 
	Thus, we can create a mapping from the open and closed edges of $\ints^2$ to the dual graph, where the edge in the dual is open if and only if the intersecting edge in $\ints^2$ is open.\\
	Then we notice that if there exists a cycle of closed edges in the dual graph enclosing the origin then the size of the open cluster at the origin is finite. 
	\begin{lemma}\label{originloop}
		$|C| < \infty \iff \exists$ a cycle of closed edges in $(\ints^2)^*$  enclosing the origin
	\end{lemma}
	\begin{proof}
		{Thinking visually since there is a ring of closed edges in the dual, the open edges from the origin in the original graph cannot extend beyond this ring. A more formal pure graph theory proof can be made but is omitted here.}
	\end{proof}
	Let $X_n = \{$There is a length n cycle of closed edges in $\ztwodual$ which surrounds the origin\}
	Then using Lemma \ref{originloop} we see
	$$\mathbb{P}(|C| < \infty) = \mathbb{P}(\bigcup_{n=4}^\infty \mathbb{P}(X_n)) \leq \sum_{n=4}^\infty \mathbb{P}(X_n) \leq \sum_{n=4}^\infty n \cdot 4(3^{n-1})q^n$$
	This sum is finite when $q<1/3$, which is when $p>2/3$. We can make the sum arbitrarily small when $p \rightarrow 1$, when the sum is smaller than 1 this implies $\theta(p) > 0$
\end{proof}
Hence the critical value $p_c \in (\frac{1}{3},1)$

\subsection {Existence of a critical value on $\ints^d$}


% \section{Including maths}

% Some maths, like $\varepsilon>0$ or $a_{23}=\alpha^3$, is written in-line. More important or complex maths is displayed on its own line.
% For example, $$ \lim_{x\to\infty}f(x)=\frac{\pi}{4}.$$

% Sometimes you need multiple lines of maths to line up nicely:

% \begin{align*}
% f(x+y)&=(x+y,-2(x+y))\\
% &=(x,-2x)+(y,-2y)\\
% &=f(x)+f(y),
% \end{align*}

% and sometimes you want to number lines in an equation

% \begin{align}
% A^{T} & =\begin{pmatrix}1 & 2\\
% 3 & 4
% \end{pmatrix}^{T}\\
% \label{my_eqn}  & =\begin{pmatrix}1 & 3\\
% 2 & 4
% \end{pmatrix}
% \end{align}

% \section{References and Figures}
% \LaTeX{} \cite{lamport94} also allows you to cite your sources. For more details on how this can be done, we refer the reader to \cite[sec:~Embedded System]{referencing}. But once you have a bibliography, you can use the cite command easily. Finally we add Figure \ref{fig:logo} to show how to add graphics. Note that we first need to make sure to have the graphic uploaded to Overleaf  or saved in the same folder as your Tex file (whichever is relevant to your case). Notice how the picture was resized using the scale command and that \LaTeX{} determine that the picture looks better above.

% \begin{figure}
%     \centering
%     \includegraphics[scale=0.6]{uob.pdf}
%     \caption{The logo for the University of Bristol}
%     \label{fig:logo}
% \end{figure}

\printbibliography
% \begin{thebibliography}{99}
% \bibitem{steif}

% % \bibitem{lamport94}
% %   Leslie Lamport,
% %   \textit{\LaTeX: a document preparation system},
% %   Addison Wesley, Massachusetts,
% %   2nd edition,
% %   1994.
  
% % \bibitem{referencing}
% %     Wikibooks,
% %     \textit{\LaTeX/Bibliography Management},
% %     [Online],
% %     Accessed at https://en.wikibooks.org/wiki/LaTeX/Bibliography\_Management,
% %     (DATE ACCESSED).
    

% \end{thebibliography}

\end{document}